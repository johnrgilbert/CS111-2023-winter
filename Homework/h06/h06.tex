\documentclass[11pt]{article} 

\usepackage{amssymb,amsmath}

\newcommand{\numpy}{{\tt numpy}}            % tt font for numpy
\newcommand{\scipy}{{\tt scipy}}            % tt font for scipy
\newcommand{\matplotlib}{{\tt matplotlib}}  % tt font for matplotlib

% \topmargin -1in
% \textheight 9in
% \oddsidemargin  -.25in
% \evensidemargin -.20in
% \textwidth 7in
\topmargin -.5in
\textheight 8in
\oddsidemargin  0in
\evensidemargin .05in
\textwidth 6.5in

\begin{document}

$$\mbox{\Large \bf CS 111: Homework 6: Due by 11:59 pm Sunday, February 26, 2023}$$
\par\smallskip\noindent
{\bf Submit your paper as one PDF file,
and tell GradeScope which page(s) each problem is on.
If you worked with a partner, you must each turn in your own 
homework paper, and report the name and perm number of your partner.
No groups of more than two allowed.
}

\par\bigskip
{\bf Background:}
In this homework you'll learn how to solve least squares problems using 
a different matrix factorization instead of the SVD.
Given a matrix $A$ with $m$ rows and $n$ columns (where $m \ge n$),
the {\em QR factorization} is
$$A = QR,$$
where $Q$ is $m$-by-$m$ and orthogonal, and $R$ is $m$-by-$n$ and upper triangular.
When $m>n$, there is also an ``economy size'' QR factorization,
in which $Q$ is $m$-by-$n$ with columns orthogonal to each other, 
and $R$ is square and upper triangular.
Please read NCM Section~5.5 to learn more about the QR factorization.

You can use QR factorization to solve $Ax=b$ when $A$ is square,
because $QRx=b$ is the same as $Rx=Q^Tb$. That's Problem 1 below.
As you'll see in Problems 2 and 3, you can also use QR factorization
to solve the least squares problem $x = \arg\min||Ax-b||_2$ when $m>n$.
In practice QR is a little less expensive than SVD for dense matrices
(by maybe a factor of 2), and it can be a lot less expensive when $A$ is sparse.
Numpy's {\tt npla.lstsq()} uses SVD, 
but most large-scale least squares computations use QR.

\par\medskip
{\bf Note:}
When we ask you to ``compare'' matrix $A$ with matrix $B$, 
we mean that you should compute and print the relative norm of their difference,
\begin{verbatim}
    npla.norm(A - B) / npla.norm(B)
\end{verbatim}
This uses the Frobenius matrix norm, which is fine for this purpose.

\par\bigskip
{\bf 1.}
Let
$$A =
   \left(
   \begin{array}{ccc}
    4 & -1 & -1 \\ 	
   -1 &  4 & -1 \\ 
   -1 & -1 &  4 \\
   \end{array} \right)
$$
and let $b = (15, -3, 12)^T$.

\par\medskip
{\bf 1.1.}
Use the {\tt scipy} QR factorization routine {\tt scipy.linalg.qr()}
to compute the two matrices (orthogonal and upper triangular) that
constitute the QR factorization of $A$.
Print $Q$ and $R$, and verify by inspection that $R$ is upper triangular.
Verify that $Q$ is orthogonal by comparing $Q^TQ$ to an identity matrix.
Verify that the factorization is correct by multiplying the factors and 
comparing the result to $A$.

\par\medskip
{\bf 1.2.}
Use {\tt cs111.Usolve()} and/or {\tt cs111.Lsolve()} to compute the solution $x$
to $Ax=b$ from the QR factors, without calling any other factorization or solve routine.
(You are allowed to transpose any matrix if you want.)
Verify that $x$ is correct by computing (and showing) the relative residual norm.
Show all the Jupyter input and output for your computations.

\par\bigskip
{\bf 2.}
In this problem you will delve into the similarities and differences 
between the ```full-size'' and ``economy-size'' QR factorizations of 
a matrix with more rows than columns.
Start by generating a random 9-by-5 matrix $A$, 
using {\tt np.random.rand()}.
Show all your work in Jupyter.

\par\medskip
{\bf 2.1.}
Use {\tt Q1, R1 = scipy.linalg.qr(A)} to generate the full-size QR
factorization of $A$.
What are the dimensions of $Q_1$? Of $R_1$?
Verify that $Q_1$ is orthogonal by comparing some matrix to an identity matrix.
Verify by inspection that $R_1$ is upper triangular; 
note what ``triangular'' means for a non-square matrix.
Verify by comparing matrices that in fact $Q_1R_1=A$.

\par\medskip
{\bf 2.2.}
Use {\tt Q2, R2 = scipy.linalg.qr(A, mode='economic')} 
to generate the economy-size QR factorization of $A$.
What are the dimensions of $Q_2$? Of $R_2$?
What is $Q_2^TQ_2$?  Is $Q_2$ orthogonal? Why or why not?
Verify by inspection that $R_2$ is upper triangular.
Verify by comparing matrices that in fact $Q_2R_2=A$.

\par\medskip
{\bf 2.3.}
In English words, what is the relationship between $Q_1$ and $Q_2$?  
Use python to compute the relative norm of a difference of
some two matrices to demonstrate that relationship.

In English words, what is the relationship between $R_1$ and $R_2$?  
Use python to compute the relative norm of a difference of
some two matrices to demonstrate that relationship.

\par\bigskip
{\bf 3.}
Here you will solve a least squares problem with your 9-by-5 matrix $A$ 
from Problem~2 above.
Use {\tt b = np.random.rand(9)} to generate a random right-hand side $b$. 
(It's important that $b$ is random here.)
The least-squares problem is
$$x = \arg\min||Ax-b||_2.$$
Since the system is overdetermined, 
we do not expect there to be any $x$ that makes the residual norm zero.
It is a theorem, though, that at the least-squares solution the residual vector
is perpendicular to every column of $A$. 
That is, if $x$ minimizes the 2-norm of the residual
$r = Ax-b$, then $A^Tr=0$ is the vector of all zeros.

To see how QR factorization can be used to solve this problem, 
think first about the full-size QR factorization from Problem 2.1, 
where $A = Q_1R_1$, with $Q_1$ orthogonal and $R_1$ rectangular and upper triangular.
Multiplying by an orthogonal matrix doesn't change the 2-norm of a vector, so
$$||Ax-b||_2 = ||Q_1^T(Ax-b)||_2 = ||Q_1^TQ_1R_1x - Q_1^Tb||_2 = ||R_1x - Q_1^Tb||_2.$$
Much like the SVD method we saw in class, we can now solve the first $n$ equations of
$R_1x=Q_1^Tb$ exactly, and nothing we do to $x$ will make any difference in the last
$m-n$ equations because $R_1$ is zero in those rows.

Now the trick is to notice that we can do the same thing with the economy-size factorization.
You'll see how in Problem 3.2 below.

\par\medskip
{\bf 3.1}
Use {\tt npla.lstsq()} to compute the least-squares solution $x$. 
Print $x$ and the relative residual norm $||b-Ax||_2/||b||_2$.
Verify that the residual is orthogonal to the columns of $A$ by 
computing (and printing) $||A^Tr||_2$.

\par\medskip
{\bf 3.2}
Use the economy-size factorization $Q_2R_2=A$ from Problem 2.2 to solve for $x$ 
as follows, showing your work in Jupyter.
First compute $y = Q_2^Tb$. 
Then solve $R_2x=y$ for $x$, using an appropriate routine from the {\tt cs111} module.
As above, print $x$ and the relative residual norm, 
and verify that the residual is orthogonal to the columns of $A$.

\par\bigskip
{\bf 4.}
Consider each of the following python loops.
For each loop, answer:
How many iterations does it do before halting?
What are the last two values of $x$ it prints
(both as decimals printed by python, and as
IEEE standard 16-hex-digit representations
as printed by {\tt cs111.print\_float64})?

For each loop, explain in one English
sentence what property of the floating-point system 
the loop's behavior demonstrates.

\par\medskip
{\bf 4.1.}
\begin{verbatim}
    x = 1.0
    while 1.0 + x > 1.0:
        x = x / 2.0
        print(x)
\end{verbatim}

\par\medskip
{\bf 4.2.}
\begin{verbatim}
    x = 1.0
    while x + x > x:
        x = 2.0 * x
        print(x)
\end{verbatim}

\par\medskip
{\bf 4.3.}
\begin{verbatim}
    x = 1.0
    while x + x > x:
        x = x / 2.0
        print(x)
\end{verbatim}
        

\end{document}
