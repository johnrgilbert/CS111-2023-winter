\documentclass[11pt]{article} 

\usepackage{amssymb,amsmath}

\newcommand{\numpy}{{\tt numpy}}            % tt font for numpy
\newcommand{\scipy}{{\tt scipy}}            % tt font for scipy
\newcommand{\matplotlib}{{\tt matplotlib}}  % tt font for matplotlib

% \topmargin -1in
% \textheight 9in
% \oddsidemargin  -.25in
% \evensidemargin -.20in
% \textwidth 7in
\topmargin -.5in
\textheight 8in
\oddsidemargin  -.25in
\evensidemargin -.20in
\textwidth 7in

\begin{document}

$$\mbox{\Large \bf CS 111: Homework 2: Due by 11:59 pm Sunday, January 29, 2023}$$
\par\smallskip\noindent
{\bf Submit your paper as one PDF file,
and tell GradeScope which page(s) each problem is on.
If you worked with a partner, you must each turn in your own 
homework paper, and report the name and perm number of your partner.
No groups of more than two allowed.
}

\par\bigskip
{\bf 1.} Consider the permutation matrix 
$$P =
   \left(
   \begin{array}{cccc}
    0 & 1 & 0 & 0 \\ 	
    0 & 0 & 0 & 1 \\ 	
    0 & 0 & 1 & 0 \\ 	
    1 & 0 & 0 & 0 \\ 	
   \end{array} \right).
$$

\par\medskip
{\bf 1.1.}
Find a 4-permutation {\tt v = [ something ]}
such that {\tt A[v,:] == P @ A} holds for
{\em every}\, 4-by-4 matrix $A$.
Test your answer by running a few lines of Python, 
and include the code and output from your testing
in your LaTeX writeup.

\par\medskip
{\bf 1.2.} For the same $P$, 
find a 4-permutation {\tt w = [ something ]}
such that {\tt A[:,w] == A @ P} holds for
{\em every}\, 4-by-4 matrix $A$.
Turn in your testing for your answer.

\par\bigskip
{\bf 2.}
The routine {\tt Lsolve()} is in the file {\tt cs111/LU.py}.
It is called as {\tt y = Lsolve(L, b)}, where $L$ must be
unit lower triangular, that is,
a square, lower triangular matrix whose diagonal is all ones.
Modify {\tt Lsolve()} to take an optional third keyword argument
{\tt unit\_diag} that defaults to {\tt True}. 
If {\tt unit\_diag} is {\tt False}, 
your modified routine should not require (or {\tt assert}) 
that the diagonal is all ones, but instead it should do
the necessary arithmetic to get the right answer to $Ly=b$
for any nonsingular lower triangular matrix $L$. 
Test your answer, both by itself and with {\tt LUsolve()},
and include a screenshot of your testing along with
your code as part of your LaTeX writeup.

\par\bigskip
{\bf 3.}
Write {\tt Usolve()}, analogous to {\tt Lsolve()} in the file {\tt cs111/LU.py},
to solve an upper triangular system $Ux=y$. 
You should again include an optional argument {\tt unit\_diag},
as in problem (2), but this time its default should be {\tt False}.
Test your answer, both by itself and with {\tt LUsolve()},
and include a screenshot of your testing along with
your code as part of your LaTeX writeup.
Hint: Loops can be run backward in Python, 
say from $n-1$ down to $0$, by writing
$$\mbox{\tt for i in reversed(range(n)):}$$

\par\bigskip
{\bf 4.}
Suppose that $A$ is a nonsymmetric invertible matrix, 
$b$ is a vector, and that you have called 
$$\mbox{\tt L, U, p = cs111.LUfactor(A)}.$$
Now suppose you want to solve the system $A^Tx=b$ (not $Ax=b$) for $x$.
Show how to do this using only calls to {\tt Lsolve()} and {\tt Usolve()}
(as modified in problems (2) and (3)).

You may not call {\tt LUsolve()} or any of {\tt numpy}'s built-in solvers (like {\tt npla.solve()}),
and you may not call {\tt LUfactor()} again.
You are allowed to transpose any matrices you wish;
recall that {\tt M.T} means the transposed matrix $M^T$ in {\tt numpy}.
Test your method in {\tt numpy} on a randomly generated 6-by-6 matrix
and show the code and output in Jupyter.

\end{document}
