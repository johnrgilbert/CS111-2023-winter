\documentclass[11pt]{article} 

\usepackage{amssymb,amsmath}

\newcommand{\numpy}{{\tt numpy}}            % tt font for numpy
\newcommand{\scipy}{{\tt scipy}}            % tt font for scipy
\newcommand{\matplotlib}{{\tt matplotlib}}  % tt font for matplotlib

% \topmargin -1in
% \textheight 9in
% \oddsidemargin  -.25in
% \evensidemargin -.20in
% \textwidth 7in
\topmargin -.5in
\textheight 7.5in
\oddsidemargin  -0.25in
\evensidemargin -0.20in
\textwidth 7in

\begin{document}

$$\mbox{\Large \bf CS 111: Homework 2: Due by 11:59 pm Sunday, October 10, 2021}$$
\par\smallskip\noindent
{\bf Submit your paper as one PDF file,
and tell GradeScope which page(s) each problem is on.
If you worked with a partner, you must each turn in your own 
homework paper, and report the name and perm number of your partner.
No groups of more than two allowed.
}

\par\bigskip
{\bf 1.}
Consider the code for the temperature problem in 
the file {\tt cs111/temperature.py}, 
especially the routines {\tt make\_A()} and {\tt make\_b()}
that create the matrix $A$ and right-hand side $b$.
Experiment with different ways of setting the boundary conditions,
which are the parameters {\tt top}, {\tt bottom}, {\tt left}, and {\tt right} 
to {\tt make\_b()}.
Make a plot of the most interesting result that you get (in your opinion), 
and explain how you got it. 
If you want, you can also experiment with {\tt matplotlib} 
to make a more interesting plot of your result. 
(The CS 111 logo on the course web page was obtained this way in 2010; 
maybe we can get a new logo this year!)

\par\bigskip
{\bf 2.}
Again consider the routines {\tt make\_A()} and {\tt make\_b()}
that create the matrix~$A$ and right-hand side~$b$ for the temperature problem.
Let $k=100$.

\par\medskip
{\bf 2.1.}
How many elements are there in $b$?

\par\medskip
{\bf 2.2.}
Considering all possible choices for the temperatures on the boundary,
what is the largest number of elements of $b$ that could possibly 
be nonzero? 

\par\medskip
{\bf 2.3.}
Explain why the rest of the elements of $b$ are zero, no matter
what the boundary temperatures are.

\par\bigskip
{\bf 3.} Consider the permutation matrix 
$$P =
   \left(
   \begin{array}{cccc}
    0 & 1 & 0 & 0 \\ 	
    0 & 0 & 0 & 1 \\ 	
    0 & 0 & 1 & 0 \\ 	
    1 & 0 & 0 & 0 \\ 	
   \end{array} \right).
$$

\par\medskip
{\bf 3.1.}
Find a 4-permutation {\tt v = [ something ]}
such that {\tt A[v,:] == P @ A} holds for
{\em every}\, 4-by-4 matrix $A$.
Test your answer by running a few lines of Python, 
and turn in the result.

\par\medskip
{\bf 3.2.} For the same $P$, 
find a 4-permutation {\tt w = [ something ]}
such that, for {\em every}\, 4-by-4 matrix $A$, 
we have {\tt A[:,w] == A @ P}. 
Test your answer and turn in the result.

\par\bigskip
{\bf 4.}
Write {\tt Usolve()}, analogous to {\tt Lsolve()} in 
the file {\tt cs111/LU.py},
to solve an upper triangular system $Ux=y$. 
Warning: Notice that, unlike in {\tt Lsolve()}, 
the diagonal elements of $U$ don't have to be equal to one.
Test your answer, both by itself and with {\tt LUsolve()},
and turn in the result.
Hint: Loops can be run backward in Python, 
say from $n-1$ down to $0$, by writing
$$\mbox{\tt for i in reversed(range(n)):}$$

\end{document}
