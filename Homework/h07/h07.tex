\documentclass[11pt]{article} 

\usepackage{amssymb,amsmath}

\newcommand{\numpy}{{\tt numpy}}            % tt font for numpy
\newcommand{\scipy}{{\tt scipy}}            % tt font for scipy
\newcommand{\matplotlib}{{\tt matplotlib}}  % tt font for matplotlib

% \topmargin -1in
% \textheight 9in
% \oddsidemargin  -.25in
% \evensidemargin -.20in
% \textwidth 7in
\topmargin -.5in
\textheight 7.5in
\oddsidemargin  -0.25in
\evensidemargin -0.20in
\textwidth 7in

\begin{document}

$$\mbox{\Large \bf CS 111: Homework 7: Due by 11:59 pm Sunday, November 14, 2021}$$
\par\smallskip\noindent
{\bf Submit your paper as one PDF file,
and tell GradeScope which page(s) each problem is on.
If you worked with a partner, 
you must each separately write up and turn in your own homework paper, 
and report the name of your partner.
No groups of more than two.
}

\par\bigskip
{\bf Background.}
In this homework you'll learn how to solve least squares problems using 
a different matrix factorization instead of the SVD.
Given a matrix $A$ with $m$ rows and $n$ columns (where $m \ge n$),
the {\em QR factorization} is
$$A = QR,$$
where $Q$ is $m$-by-$m$ and orthogonal, and $R$ is $m$-by-$n$ and upper triangular.
When $m>n$, there is also an ``economy size'' QR factorization,
in which $Q$ is $m$-by-$n$ with orthogonal columns, 
and $R$ is square and upper triangular.
Please read NCM Section~5.5 to learn more about the QR factorization.

You can use QR factorization to solve $Ax=b$ when $A$ is square,
because $QRx=b$ is the same as $Rx=Q^Tb$. That's Problem 1 below.
As you'll see in Problems 2 and 3, you can also use QR factorization
to solve the least squares problem $\min_x||Ax-b||_2$ when $m>n$.
In practice QR is a little less expensive than SVD for dense matrices
(by maybe a factor of 2), and it can be much less expensive when $A$ is sparse.
Numpy's {\tt npla.lstsq()} uses SVD, 
but most large-scale least squares computations use QR.

\par\bigskip
{\bf 1.}
Let
$$A =
   \left(
   \begin{array}{ccc}
    4 & -1 & -1 \\ 	
   -1 &  4 & -1 \\ 
   -1 & -1 &  4 \\
   \end{array} \right)
$$
and let $b = (15, -3, 12)^T$.

\par\medskip
{\bf 1.1.}
Use the {\tt scipy} QR factorization routine {\tt scipy.linalg.qr()}
to compute the two matrices (orthogonal and upper triangular) that
constitute the QR factorization of $A$.
Print $Q$ and $R$, and verify by inspection that $R$ is upper triangular.
Verify that $Q$ is orthogonal by comparing $Q^TQ$ to the identity matrix.
Verify that the factorization is correct by multiplying the factors and 
comparing the result to $A$.

Note: When we say to ``compare'' one matrix $A$ to another matrix $B$, 
it is sufficient to compute (and print) the relative norm of their difference,
\begin{verbatim}
    npla.norm(A - B) / npla.norm(B)
\end{verbatim}
Which matrix norm to use? 
As you can see by saying {\tt npla.norm?}, 
the default matrix norm in {\tt numpy} is the {\em Frobenius norm}, 
which is the square root of the sum of the squares 
of all $n^2$ elements of the matrix.
The Frobenius norm is a good one to use for comparing two matrices, 
because it's much easier to compute than the 2-norm,
but it isn't as useful in the analysis of matrix algorithms.

\par\medskip
{\bf 1.2.}
Use {\tt cs111.Usolve()} and/or {\tt cs111.Lsolve()} to compute the solution $x$
to $Ax=b$ from the QR factors, without calling any other factorization or solve routine.
(You are allowed to transpose any matrix if you want.)
Verify that $x$ is correct by computing (and showing) the relative residual norm.
Show all the Jupyter input and output for your computations.

\par\bigskip
{\bf 2.}
In this problem you will delve into the similarities and differences 
between the ```full-size'' and ``economy-size'' QR factorizations of 
a matrix with more rows than columns.
Start by generating a random 9-by-5 matrix $A$, 
using {\tt np.random.rand()}.
Show all your work in Jupyter.

\par\medskip
{\bf 2.1.}
Use {\tt Q1, R1 = scipy.linalg.qr(A)} to generate the full-size QR
factorization of $A$.
What are the dimensions of $Q_1$? Of $R_1$?
Verify that $Q_1$ is orthogonal by comparison to an identity matrix.
Verify by inspection that $R_1$ is upper triangular; 
note what ``triangular'' means for a non-square matrix.
Verify that in fact $Q_1R_1=A$.

\par\medskip
{\bf 2.2.}
Use {\tt Q2, R2 = scipy.linalg.qr(A, mode='economic')} 
to generate the economy-size QR factorization of $A$.
What are the dimensions of $Q_2$? Of $R_2$?
What is $Q_2^TQ_2$?  Is $Q_2$ orthogonal? Why or why not?
Verify by inspection that $R_2$ is upper triangular.
Verify that in fact $Q_2R_2=A$.

In English words, what is the relationship between $Q_1$ and $Q_2$?  
Use python to compute the relative Frobenius norm of a difference of
two matrices to demonstrate that relationship.

In English words, what is the relationship between $R_1$ and $R_2$?  
Use python to compute the relative Frobenius norm of a difference of
two matrices to demonstrate that relationship.

\par\bigskip
{\bf 3.}
Here you will solve a least-squares approximation problem with 
your 9-by-5 matrix $A$ from Problem~2 above.

Begin by using {\tt b = np.random.rand(9)} to generate a random
right-hand side $b$. 
(It's important for this experiment that $b$ is random.)
What does it mean to look for a solution to $Ax=b$ in this case?
We would need a 5-vector $x$ that satisfies all 9 equations 
represented by the rows of $A$, 
but such a vector (almost certainly) doesn't exist because the
linear equation system is {\em overdetermined}.
Instead, we will seek the {\em least-squares} solution, 
which is the 5-vector $x$ that minimizes the 2-norm of the 
residual $b-Ax$.
The least-squares problem is
$$\min_x||b-Ax||_2.$$

Since the system is overdetermined, 
we do not expect there to exist an $x$ that makes the residual norm zero.
It is a theorem, though, that for the least-squares solution the residual vector
is orthogonal (perpendicular) to every column of $A$. 
That is, if $x$ is the value that achieves the smallest possible 
residual $r = b - Ax$, as measured by the 2-norm,
then $A^Tr=0$ is the vector of all zeros.

\par\medskip
{\bf 3.1}
Use {\tt npla.lstsq()} to compute the least-squares solution $x$. 
Print $x$ and the relative residual norm $||b-Ax||_2/||b||_2$.
Verify that the residual is orthogonal to the columns of $A$ by 
computing (and printing) $||A^Tr||_2$.

\par\medskip
{\bf 3.2}
Use the full-size factorization $Q_1R_1=A$ from Problem 2 to solve for $x$ 
as follows, showing your work in Jupyter.
First compute $y = Q_1^Tb$. 
Then ``solve'' $R_1x=y$ for $x$,
which will require extracting a submatrix of $R_1$ before calling 
{\tt cs111.Usolve()}.
As above, print $x$ and the relative residual norm, 
and verify that the residual is orthogonal to the columns of $A$.

\par\medskip
{\bf 3.3}
Use the economy-size factorization $Q_2R_2=A$ from Problem 2 to solve for $x$ 
as follows, showing your work in Jupyter.
First compute $y = Q_2^Tb$. 
Then solve $R_2x=y$ for $x$.
Notice that this time $R_2$ is square, so you can just call
{\tt cs111.Usolve()}.
Why does this give the same result?
As above, print $x$ and the relative residual norm, 
and verify that the residual is orthogonal to the columns of $A$.

\par\medskip
{\bf 3.4}
Finally, change the right-hand side $b$ to be equal to the row sums
of $A$ (by multiplying by the appropriate vector of 1s).
Use any of the methods above (your choice) to compute the new
least-squares solution $x$. What is the relative residual now?
What is unusual about the new relative residual, and why did this happen?
Explain in English.

\end{document}
